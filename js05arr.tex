%!TEX program = xelatex
\documentclass{article}
\usepackage[a5paper,hmargin=17mm,tmargin=15mm,bmargin=25mm]{geometry}

\usepackage{ifxetex}
\ifxetex
 \usepackage{fontspec}
 \setmainfont[Scale=1.1]{Arno Pro}
 \setmonofont[Scale=.8]{Lucida Console} %% consolas ломает знаки минуса
 \usepackage{unicode-math}              %% пакет для загрузки шрифтов математического режима 
 \setmathfont{[latinmodern-math.otf]}
 \setmathfont[range=\mathit/{latin,Latin}]{Arno Pro Italic}
 \setmathfont[range=up]{Arno Pro}
\else
 \usepackage[utf8]{inputenc}
\fi
\usepackage[russian]{babel}
\usepackage{enumitem, graphicx, minted}


\begin{document}
\section*{{\normalsize Лабораторная работа 5} \\ Массивы}

\noindent\vskip-18mm
\null\hfill\includegraphics[scale=.13]{logo.png}
\newline

{\bf Пример 1}.~Напишите функцию \texttt{f(A)}, принимающую массив действительных чисел $A$ = [$a_0$, $a_1$, $\ldots$, $a_{n-1}$] и меняющую в массиве местами первый отрицательный (если он есть) и последний элементы массива.

\smallskip\noindent\textbf{Решение:}

\begin{minted}{js}
function swapLastAndFirstNeg(A) {
    let i, n, buf = A.length;
    for (i = 0; i < n; i++) {
       if (A[i] < 0) break;
    }
    if (i < n){
        buf = A[i];
        A[i] = A[n-1];
        A[n-1] = buf;
    }
}
\end{minted}

\noindent Почему условие наличия отрицательных элементов выглядит как $i<n$? Почему для перестановки элементов будет неправильно написать следующие присваивания?
\begin{minted}{js}
        A[i] = A[n-1]; 
        A[n-1] = A[i];
\end{minted}

{\bf Пример 2}. Напишите функцию \texttt{f(A)}, принимающую массив действительных чисел $A$ = [$a_0$, $a_1$, $\ldots$, $a_{n-1}$] и возвращающую новый массив, полученный из $A$ обменом первого отрицательного и последнего элемента массива.

\smallskip\noindent\textbf{Решение:}
\begin{minted}{js}
function swappedLastAndFirstNeg(A) {
    let i, firstneg = -1, n = A.length;
    B = Array(n);
    for (i = 0; i < n-1; i++) {
        if (A[i] < 0 && firstneg === -1) {
            firstneg = i;
            B[i] = A[n-1];
        } else {
            B[i] = A[i];
        }
    }
    if (firstneg >= 0){
        B[n-1] = A[firstneg];
    } else {
        B[n-1] = A[n-1];
    }
    return B;
}
\end{minted}

{\bf Пример 3}.~Напишите функцию \texttt{f(A)}, принимающую массив действительных чисел $A$ = [$a_0$, $a_1$, $\ldots$, $a_{n-1}$]. Функция должна менять местами первый отрицательный и последний элемент массива $A$. Функция должна возвращать индекс первого отрицательного элемента, либо \texttt{null}, если отрицательных элементов в массиве нет. 

\smallskip\noindent\textbf{Решение:}
\begin{minted}{js}
function getSwappedLastAndFirstNeg(A) {
    let i, n = A.length, buf;
    for (i = 0; i < n; i++) {
        if (A[i] < 0) {
            break;
        }
    }
    if (i < n){
        buf = A[i];
        A[i] = A[n-1];
        A[n-1] = buf;
        return i;
    } 
    return null;
}
\end{minted}


{\bf Пример 4}.~{\bf Бинарный поиск}. Напишите функцию \texttt{findPos(A, p)}, принимающую массив действительных чисел $A = [a_1, a_2, \, \ldots]$, упорядоченных по возрастанию и без повторов: $a_i \neq a_{i+1}$. Функция возвращает индекс элемента массива, равного $p$, или \texttt{null}, если число $p$ не встречается среди $a_i$. Метод \texttt{indexOf} не использовать.

\pagebreak[4]
\noindent
\medskip\noindent\textbf{Решение:}
\begin{minted}{js}
function findPos(A, p) {
    let left, right, mid, n = A.length;
    // инвариант: A[right],A[right+1],... - больше p
    //            ....A[left-2],A[left-1] - меньше или равно p
    left = 0; right = n;
    while (left < right) {
        mid = Math.floor((left+right)/2);
        if (A[mid] == p) return mid;
        if (A[mid] > p) {
            right = mid;
        } else {
            left = mid+1;
        }
    }
    return null;
}
\end{minted}
Тут стоит хорошо подумать про то, почему выписанное в комментарии утверждение действительно инвариант, почему в условии цикла стоит \texttt{<}, а не \texttt{<=},
почему в \texttt{else} присутствует +1, и почему мы обязательно найдем элемент, равный $p$, если он есть. 

\newpage

\subsection*{ОБЩИЕ ЗАДАЧИ}
\def\twodigit#1{% 
\ifnum#1<10 0\fi 
\number#1}

Решения и тесты общих задач должны быть названы \texttt{arr-01.js}, \texttt{run-01.js}, \texttt{arr-02.js}, \texttt{run-02.js}, и т.~д.

\begin{enumerate}[label=0.\theenumi.]
\item 
На почте выстроилась очередь из $n$ клиентов. Время обслуживания $i$-го клиента равно $t_i$. Напишите функцию \texttt{f(T)}, которая возвращает массив $W$, где  $w_i$~--- время ожидания $i$-го клиента в очереди.
\\Пример: \texttt{f([7, 10, 15, 22])} равно \texttt{[0,7,17,32]}.

\item 
Напишите функцию \texttt{f(A)}, принимающую массив действительных чисел, и возвращающую массив из трех чисел: количества отрицательных, положительных и нулевых элементов в указанном порядке. 
\\Пример: \texttt{f([9,-5,0,-2,8,-1])} равно \texttt{[3,1,2]}.

\item 
У вас есть $x$ биткойнов. Вы хотите избавиться от них, обменяв их на рубли. Информация о курсе обмена на нескольких надежных онлайн биржах дана в виде массива $K = [K_1,k_1,K_2,k_2,\ldots,K_n,k_n]$, где $K_i$, $k_i$~--- соответственно, стоимость покупки и продажи биткойна на $i$-й онлайн бирже. Составьте функцию \texttt{f(K,x)}, определяющую, какое максимальное количество рублей вы получите, продав свои биткойны на одной из бирж.
\\Пример: \texttt{f([520668,524853,525000,530000],0.001)} равно \texttt{525}.

\item 
Напишите функцию \texttt{oddMax(A)}, возвращающую максимальный нечетный элемент в массиве натуральных чисел $A$, или \texttt{undefined}, если в нем нет нечетных элементов.
\\Пример: \texttt{oddMax([7, 10, 15, 22])} равно \texttt{15}.
\end{enumerate}

\newpage

\begin{center}
\subsection*{Индивидуальные задания}
\end{center}

\noindent \textbf{Всего в этом домашнем задании 5 пунктов. Из каждого пункта надо сделать по одной задаче, в зависимости от вашего номера в списке группы. Считать по кругу: если в пункте 6 задач, то 7-й номер делает первую задачу, 8-й номер вторую и т.д.}
{
    \color{red}
    \bf
    \\[1.5mm]
    Нельзя пользоваться никакими свойствами и методами массивов, кроме \texttt{length} и \texttt{push()} (никаких \texttt{indexOf}, \texttt{slice}, и т.\,п.).
    \\[1.5mm]
    Баллы за задачу ставятся только при соблюдении следующих требований:
}


\medskip

\begin{enumerate}[nolistsep]
    \item
    Все функции сохранять по отдельности в файлах с именами \texttt{arr-$nn$.js}, где $nn$~--- номер задачи без точки, например, решение задачи 3.6 сохранять в файле \texttt{arr-36.js}.
    \item
    В начало файла решения вставлять в виде комментария условие задачи. 
    \item
    Функцию переименовать из $f$ в подходящую по смыслу. 
    \item
    В конце файла добавлять строку
    \\[.7mm]
    \texttt{module.exports = \textit{имяВашейФункции};}
    \item
    Создать файл тестов с названием \texttt{run-$nn$.js}, с тем же номером~$nn$, содержащий строки

    \smallskip
    \verb!   const f = require('./arr-!$nn$\verb!.js');!\\
    \verb!   const assert = require('assert');!

    \smallskip
    \noindent
    а затем не менее 3 проверок вида 
    
    \smallskip
    \verb!   assert.deepStrictEqual(..., ..., " ... ");!

    \smallskip
    \noindent
    где первый аргумент~--- что проверяется, второй~--- чему оно должно быть равно,
    третий~--- строка, которая будет напечатана, если сопадения нет, например, для задания 1.1
    проверка примера из условия выглядит как

    \smallskip
    \verb!   assert.deepStrictEqual(f([9,5,6,2,8]), 7,!\\
    \verb!                          "ОШИБКА ДЛЯ [9,5,6,2,8]");!

\end{enumerate}





\subsubsection*{Задачи}
\begin{enumerate}[label={}, leftmargin=0pt, itemindent=0pt]
\item

% 1. Результат --- одно значение
\begin{enumerate}[label=\arabic{enumi}.\arabic*.]
\item 
Напишите функцию \texttt{f(A)}, принимающую последовательность действительных чисел $a_0$, $a_1$, $\ldots$, $a_{n-1}$ в виде массива и возвращающую наименьшую длину отрезка числовой оси, содержащего все эти числа.
\\Пример: \texttt{f([9,5,6,2,8])} равно \texttt{7}.

\item 
При поступлении в вуз абитуриенты, получившие «двойку» на первом экзамене, ко второму не допускаются. В массиве $A$ записаны оценки экзаменующихся, полученные на первом экзамене. Напишите функцию \texttt{f(A)}, которая подсчитает, сколько человек допущено ко второму экзамену.
\\Пример: \texttt{f([3,2,5,4,5,3,5,2])} равно \texttt{6}.

\item 
Функция \texttt{f(A)} принимает последовательность действительных чисел $a_0$, $a_1$, $\ldots$, $a_{n-1}$, в которой есть только положительные и отрицательные элементы, вычисляет произведение отрицательных элементов $P_1$ и произведение положительных элементов $P_2$, и возвращает большее по модулю из них.
\\Пример: \texttt{f([2,-3,-5,7,-3,3])=-45}. 

\item 
Напишите функцию \texttt{f(A, M)}, которая принимает массив целых положительных чисел и возвращает произведение только тех элементов $A$, которые больше заданного числа $M$. Если таких нет, то функция возвращает $1$.
\\Пример: \texttt{f([3,2,5,4,5,3,5,2],\,4)} равно \texttt{125}.

\item 
Назовем серией последовательность одинаковых подряд идущих элементов массива. 
Напишите функцию \texttt{f(A)}, которая находит наиболее в массиве $A$ число, образующее максимально длинную серию. Если таких чисел несколько, то функция возвращает наименьшее из них.
\\Пример: \texttt{f([7,7,7,3,2,1,1,5,5,5]) = 5}.

\item 
Назовем локальным максимумом элемент массива, который не меньше своих соседей (у крайних эелементов~--- только один сосед, у внутренних~--- по два). Напишите функцию, которая возвращает число локальных максимумов в массиве.
\\Пример: \texttt{f([2,3,4,5,5,3,1,0,2]) = 3}.

\item Напишите функцию \texttt{f(A,m)}, которая принимает массив целых чисел $a_i$ и дает число пар индексов $i$, $j$, таких, что $i<j$ и $a_i + a_j=m$.
\\Пример: \texttt{f([1,2,9,8],\,10) = 2}

\item  
Напишите функцию \texttt{f(A)}, которая принимает массив из $n$ различных целых чисел, и возвращает сумму элементов массива, расположенных между максимальным и минимальным значениями (в сумму включить и оба этих числа).
\\Пример: \texttt{f([-2,2, 4,1, -3,3]) = 2}

\end{enumerate}

\hrulefill


% 2. Результат --- новый массив 
\item
\begin{enumerate}[label=\arabic{enumi}.\arabic*.]
\item 
Напишите функцию \texttt{f(A)}, которая принимает массив целых чисел $a_0$, $a_1$, $\ldots$, $a_{n-1}$ и возвращает новый массив, образованный выбрасыванием из исходного тех элементов, которые равны $\min A$.
\\Пример: \texttt{f([-2,-1,5,3,-2,7])} равно \texttt{[-1,5,3,7]}.

\item 
Напишите функцию \texttt{f(A)}, принимающую массив действительных чисел $a_0$, $a_1$, $\ldots$, $a_{n-1}$, и возвращающую новый массив (возможно, пустой) из только тех чисел $a_k$, для которых выполняется $a_k<k$.
\\Пример: \texttt{f([3,0.5,3.1,2,6])} равно \texttt{[0.5,\,2]}.

\item 
Напишите функцию \texttt{f(A, m, r)}, принимающую массив натуральных чисел $a_0$, $a_1$, $\ldots$, $a_{n-1}$, и натуральные числа $M$, $L$ ($0 \le L \le M-1$), и возвращающую новый массив (возможно, пустой), состоящий только из тех чисел, у которых  остаток от деления на $M$ равен $L$.
\\Пример: \texttt{f([3,2,7,4,6,2,9,1],\,3,\,1)} равно \texttt{[7,4,1]}.

\item 
Напишите функцию \texttt{f(A)}, которая принимает упорядоченный по неубыванию массив из $n$ различных целых чисел, находит в этом массиве минимальный $m$ и максимальный элемент $M$, и возвращает новый массив, содержащий в порядке возрастания все целые числа из интервала ($m, M$), которые не входят в данный массив.
\\Пример: \texttt{f([-2, 0, 0, 1, 3, 5]) = [-1, 2, 4]}

\item 
Напишите функцию \texttt{f(A)}, которая принимает числовой массив, в котором размещены: в первой половине значения аргумента, во второй~--- соответствующие им в том же порядке значения функции. Функция должна вернуть новый массив, в котором чередуются значения аргумента и соответствующие им значения функции.
\\Пример: \texttt{f([2,3,4,4,9,16])} равно \texttt{[2,4,3,9,4,16]}.

\item 
Пригодность детали оценивается по размеру, который должен находиться в интервале ($d - \frac{\varepsilon}{2}, d + \frac{\varepsilon}{2}$). Напишите функцию \texttt{(A,d,eps)}, которая поможет находить в партии бракованные детали, а именно возвращает в виде массива индексы тех элементов $A$, которые выходят за пределы заданного интервала.  Если таких нет, функция возвращает пустой массив.
\\Пример: \texttt{f([1.8,2.1,2.0,1.9,2.3],\,2,\,0.4)} равно \texttt{[0,4]}.

\item 
Напишите функцию \texttt{f(A,c,d)}, которая принимает массив действительных чисел $a_0$, $a_1$, $\ldots$, $a_{n-1}$, и возвращает новый массив, который содержит только те его элементы, которые принадлежат отрезку [$c, d$].
\\Пример: \texttt{f([9,5,6,2,8],\,6,\,9)} равно \texttt{[9,6,8]}.

\item 
Напишите функцию \texttt{f(A,x)}, которая принимает массив целых чисел и действительное число $x$, и возвращает новый массив, содержащий два числа из  $A$, среднее арифметическое которых ближе всего к $x$. Числа в массиве должны идти в порядке возрастания индексов. Если таких пар несколько, функция возвращает пару, у которой индекс первого элемента меньше. 
\\Пример: \texttt{f([1, 5, 0, 2, 7, 4, 4], 2.5) = [1, 4]}

\item 
Напишите функцию \texttt{f(A,k)}, которая принимает массив натуральных чисел и целое $k$,  $0\leqslant k \leqslant 9$, и возвращает новый массив, состоящий из тех элементов исходного, которые оканчиваются на цифру $k$.
\\Пример: \texttt{f([22, 35, 41, 57, 62], 2) = [22, 62]}.
\end{enumerate}

\hrulefill

% Раздел 3. Замены в переданном массиве
\item
\begin{enumerate}[label=\arabic{enumi}.\arabic*.]
\item 
Напишите функцию \texttt{f(A,b)}, которая принимает массив действительных чисел, содержащий неубывающую последовательность $a_0 \leqslant a_1 \leqslant \ldots$ и действительное число $b\geqslant a_0$, и заменяет в массиве один элемент на $b$ так, чтобы последовательность осталась неубывающей. Функция ничего не возвращает.
\\Пример: для \texttt{A = [7, 10, 15, 22]} после вызова \texttt{f(A,9)} станет \texttt{A = [7, 9, 15, 22]}.

\item 
Напишите функцию \texttt{f(A,b)}, которая принимает массив действительных чисел, содержащий строго убывающую последовательность $a_0 > a_1 > \ldots > a_{n-1}$ и действительное число $b<a_0$, которого нет среди $a_i$, и заменяет в массиве один элемент на $b$ так, чтобы последовательность осталась строго убывающей. Функция ничего не возвращает.
\\Пример: для \texttt{A = [22, 20, 15, 11, 7]} после вызова \texttt{f(A, 18)} станет \texttt{A = [22, 20, 18, 11, 7]}.

\item 
Напишите функцию \texttt{f(A,\,m)}, принимающую массив действительных чисел $A$ = [$a_0$, $a_1$, $\ldots$, $a_{n-1}$] и число $m$. Заменить все элементы массива, большие $m$, этим числом. Вернуть количество замен.
\\Пример: \texttt{f(A, 6)}, где \texttt{A=[9,5,6,2,8]} дает \texttt{2} и изменяет массив $A$, так, что он станет равен \texttt{[6,5,6,2,6]}.

\item 
Напишите функцию \texttt{f(A)}, которая принимает массив попарно различных действительных чисел и меняет в нем местами наибольший и наименьший элементы в массиве $A$. Функция ничего не возвращает.
\\Пример: для $A=$\texttt{[9, 5, 6, 2.3, 8]}, \texttt{f(A)} изменяет массив $A$, так что он станет равен \texttt{[2.3, 5, 6, 9, 8]}.



\item 
Напишите функцию \texttt{f(A)}, которая поменяет в массиве $A$ местами элемент с индексом 0 с элементом, имеющим индекс 1, второй с третьим и т.\,д. Функция ничего не возвращает.
\\Пример: для $A=$\texttt{[9,5,6,2,8]}, \texttt{f(A)} изменяет массив $A$, так что он станет равен \texttt{[5,9,2,6,8]}.

\item 
Напишите функцию \texttt{f(A)}, которая принимает массив целых чисел $a_i$ и заменяет в нем минимальный элемент целой частью среднего арифметического всех элементов. Если в последовательности несколько минимальных элементов, то заменить только последний. Функция ничего не возвращает.
\\Пример: для \texttt{A = [22, 20, 15, 6, 11]} после вызова \texttt{f(A)} станет \texttt{A = [22, 20, 15, 14, 11]}.

\item 
Напишите функцию \texttt{f(A)}, которая принимает массив действительных чисел, и заменяет в нем все неположительные числа минимальным положительным. Функция ничего не возвращает.
\\Пример: массив \texttt{A = [10, 0, -1.5, 2, 17, -3]} после вызова \texttt{f(A)} станет равен \texttt{A = [10, 2, 2, 2, 17, 2]}.


\item 
Напишите функцию \texttt{f(A)}, которая принимает массив действительных чисел, и заменяет в нем все числа, большие среднего арифметического $m$, этим значением $m$. Функция ничего не возвращает.
\\Пример: массив \texttt{A = [10, 0, -1.5, 2, 17]} после вызова \texttt{f(A)} станет равен \texttt{A = [17, 2, -1.5, 0, 10]}.

\item 
Напишите функцию \texttt{f(A)}, которая принимает массив действительных чисел, и обращает порядок элементов в нем. Функция ничего не возвращает.
\\Пример: массив \texttt{A = [10, 0, -1.5, 2, 17]} после вызова \texttt{f(A)} станет равен \texttt{A = [17, 2, -1.5, 0, 10]}.
\end{enumerate}


\hrulefill
% Раздел 4 - геометрия
\item
\begin{enumerate}[label=\arabic{enumi}.\arabic*.]
\item 
Напишите функцию \texttt{f(A)}, которая принимает массив $A$, содержащий числа $X_1, Y_1, \ldots , X_n, Y_n$, где $(X_i, Y_i)$~--- координаты $i$-й точки на плоскости. Функция возвращает массив из двух номеров.  Это номера той пары точек, расстояние между которыми наибольшее (гарантируется, что такая пара единственная, точки нумеруются с 1, сначала в массиве должен идти меньший номер).
\\Пример: \texttt{f([0,0, 3,0, 0,4]) = [2,3]}

\item 
В массиве $A$ с четным количеством элементов находятся координаты точек плоскости. Они располагаются в следующем порядке: $x_1,y_1$, $x_2,y_2$ , $x_3,y_3$ и т.\,д. Напишите функцию $f(A)$, которая находит наименьшую возможную сторону квадрата с центром в начале координат и со сторонами, параллельными осям координат, который бы содержал все эти точки.
\\Пример: \texttt{f([1,5, 0,4]) = 10}.

\item 
В массиве $A$ с четным количеством элементов находятся координаты точек плоскости. Они располагаются в следующем порядке: $x_1,y_1$, $x_2,y_2$ , $x_3,y_3$ и т.\,д. Напишите функцию $f(A)$, которая находит и возвращает минимальную толщину кольца с центром в начале координат, которое содержит все точки. Толщина кольца определяется как разность его наружного и внутреннего радиусов.
\\Пример: \texttt{f([3,4, 6,-8]) = 5}

\item 
В массиве $A$ с четным количеством элементов находятся координаты точек плоскости. Они располагаются в следующем порядке: $x_1,y_1$, $x_2,y_2$ , $x_3,y_3$ и т.\,д. Напишите функцию $f(A)$, которая находит и возвращает в виде массива номера точек (считая с 1), которые могут являться вершинами квадрата, в порядке возрастания номеров. Если таких наборов несколько, функция возвращает первый в лексикографическом порядке. Если таких наборов нет, функция должна вернуть пустой массив. При расчетах считать равными отрезки, длины которых отличаются менее чем на $10^{-5}$.
\\Пример: \texttt{f([0,0, 1,1, 2,0, 2,0, -1,1, 1,-1]) = [1,2,3,5]}

\item 
В массиве $A$ с четным количеством элементов находятся координаты точек плоскости. Они располагаются в следующем порядке: $x_1,y_1$, $x_2,y_2$ , $x_3,y_3$ и т.\,д. Напишите функцию $f(A)$, которая находит и возвращает в виде массива номера точек (считая с 1), которые могут являться вершинами равнобедренного треугольника, в порядке возрастания номеров. Если таких наборов несколько, функция возвращает первый в лексикографическом порядке. Если таких наборов нет, функция должна вернуть пустой массив.  При расчетах считать равными отрезки, длины которых отличаются менее чем на $10^{-5}$.
\\Пример: \texttt{f([0,0, 1,1, 2,0, 2,0, -1,1, 1,-1]) = [1,2,3]}

\item 
В массиве $A$ с четным количеством элементов (не менее 4) находятся координаты точек плоскости. Они располагаются в следующем порядке: $x_1,y_1$, $x_2,y_2$ , $x_3,y_3$ и т.\,д. Напишите функцию $f(A)$, которая находит и возвращает в виде массива четыре числа, из которых первые два номера (считая с 1, в порядке возрастания) самых удаленных друг от друга точек, а последние два~--- наименее удаленных друг от друга точек. Если пар с минимальным или максимальным расстоянием несколько, функция должна давать пару в которой меньший номер меньше.
\\Пример: \texttt{f([0,0, 0,3, 4,0]) = [1,2,1,3]}

\item В массиве с четным количеством элементов ($2N$) находятся координаты $N$ точек плоскости. Они располагаются в следующем порядке: $x_1,y_1$, $x_2,y_2$ , $x_3,y_3$ и т.\,д. Напишите функцию, $f(A)$, которая возвращает номера трех точек, которые являются вершинами треугольника, для которого разность числа точек вне его и внутри его является минимальной.
\end{enumerate}

\hrulefill
% Раздел 5 - эффективность
\item
\begin{enumerate}[label=\arabic{enumi}.\arabic*.]
\item 
Напишите функцию \texttt{f(A)}, принимающую массив целых чисел $a_0$, $a_1$, $\ldots$, $a_{n-1}$, и возвращающую новый массив из только тех чисел $a_k$, модули которых ~--- простые числа.
\\Пример: \texttt{f([-3,5,1,9,21,11]) = [-3,5,11]}.

\item 
Каждый солнечный день улитка, сидящая на дереве, поднимается вверх на $2$ см, а каждый пасмурный день опускается вниз на $1$~см. В начале наблюдения улитка находилась в $A$ см от земли на $B$-метровом дереве. Напишите функцию \texttt{f(A, B, D)}, которая принимает числа $A$ и $B$, а также $30$-элементный массив $D$, содержащий сведения о том, был ли соответствующий день наблюдения пасмурным (0) или солнечным (1), и возвращает высоту положения улитки к концу $30$-го дня наблюдения. Под землю улитка не опускается и летать не умеет.
\\Пример: \texttt{f(11, 20, [0,0,0,0,0,0,...,0]) = 0}.

\item 
Напишите функцию \texttt{f(A,B)}, которая принимает два числовых массива, элементы которых образуют неубывающие последовательности, и возвращает новый массив, содержащий все элементы обоих в виде неубывающей последовательности. Число действий должно быть пропорционально суммарной длине массивов.
\\Пример: \texttt{f([1,5],[2,3,4,5]) = [1,2,3,4,5,5]}.

\item 
Напишите функцию \texttt{f(A)}, которая принимает массив $A = [a_0, a_1, \ldots, a_{n-1}]$ ненулевых целых чисел. Переставить, сохраняя порядок, все положительные элементы этой последовательности в ее начало, а отрицательные --- в конец. Функция ничего не возвращает. 
\\Пример: \texttt{f([1,2,-5,2,-2])} равно \texttt{[1,2,2,-5,-2]}.

\item 
Напишите функцию \texttt{f(A,B)}, которая принимает два массива натуральных чисел. Предполагается, что в каждом из них все элементы попарно различны и упорядочены по возрастанию. Функция возвращает \texttt{true}, если все элементы второй последовательности входят в первую последовательность, и \texttt{false} в противном случае. Число действий должно быть пропорционально суммарной длине массивов.
\\Пример: \texttt{f([1,5,10,14], [1,2,5]) = false}, \newline \texttt{f([1,5,10,14], [5,14]) = false}.

\item 
Напишите функцию \texttt{f(A,k)}, которая принимает массив $A = [a_0, a_1, \ldots, a_{n-1}]$ и переставляет первые $k$ его элементов с оставшимися. Функция ничего не возвращает. Дополнительных массивов не использовать. Число действий должно быть пропорционально длине массива.
\\Пример: для $A=$\,\texttt{[9,5,6,2,8]}, \texttt{f(A,2)} изменяет массив $A$, так что он станет равен \texttt{[6,2,8,9,5]}.
\end{enumerate}

\end{enumerate}

\end{document}
